\documentclass[11pt]{article}
\usepackage{amsmath, amssymb, amsthm, mathtools, geometry, enumitem}
% hyperref provides \texorpdfstring used in section titles for PDF bookmarks
\usepackage{hyperref}
\geometry{margin=1in}

\newtheorem{theorem}{Theorem}[section]
\newtheorem{lemma}[theorem]{Lemma}
\newtheorem{proposition}[theorem]{Proposition}
\newtheorem{corollary}[theorem]{Corollary}
\theoremstyle{definition}
\newtheorem{definition}[theorem]{Definition}
\newtheorem{remark}[theorem]{Remark}

\numberwithin{equation}{section}

\begin{document}

\title{Quantized Edge Phases and Torsion on the FCC Nearest-Neighbour Complex}
\author{Francis W. Han \\ \small Ylem Invest}
\date{\today}
\maketitle

\begin{abstract}
We consider the nearest-neighbour graph $G=(V,E)$ of the face-centered cubic (FCC) lattice,
equipped with a natural $2$--dimensional cellular structure given by triangular and quadrilateral
faces. On the cellular chain groups 
\[
C_2 = \mathbb Z^F,\qquad C_1 = \mathbb Z^E,
\]
we assume a phase assignment $\Phi : C_1 \to \mathbb R/2\pi\mathbb Z$ whose sum on the boundary
of every $2$-cell is trivial modulo $2\pi$. This induces a group homomorphism on the quotient
$A = C_1 / \operatorname{im}\partial_2$, and we show that for a fixed edge $e\in E$ its class
$[e]\in A$ is a torsion element of exact order $12$. As a consequence, any $2$--chain
$Z\in C_2$ satisfying $\partial_2 Z = k e$ must have $k$ divisible by $12$, yielding a
quantization condition for the phase $\phi_e$ along $e$:
\[
\phi_e \in \frac{2\pi}{12}\mathbb Z \quad (\bmod 2\pi).
\]
We give a purely local, integer linear algebra proof of the relation $12e\in\operatorname{im}\partial_2$,
together with a complementary lower bound based on phase cocycles and a simple
least common multiple argument involving $3$-- and $4$--cycles.
\end{abstract}

\newpage

\tableofcontents

\newpage

\section{Setup and Notation}

\subsection{The FCC nearest-neighbour complex}

Let $G=(V,E)$ denote the nearest-neighbour graph of the FCC lattice in $\mathbb R^3$.
We assume a fixed (periodic) $2$-dimensional cellular structure on $G$ whose $2$-cells are:
\begin{itemize}[nosep]
  \item triangular faces (denoted $\Delta$), coming from tetrahedral cells, and
  \item quadrilateral faces (denoted $Q$), coming from octahedral cells, chosen as diagonal
        quadrilateral loops.
\end{itemize}
We denote by $F$ the set of $2$-cells. The cellular chain groups are
\[
C_2 = \mathbb Z^F,\qquad C_1 = \mathbb Z^E,
\]
equipped with the boundary map
\[
\partial_2 : C_2 \to C_1.
\]

Fix an edge $e \in E$. Its local star $\operatorname{St}(e)$ consists of the $2$-cells incident
on $e$. In the FCC complex, there are precisely six such faces:
\[
(Q_1,Q_2,\Delta_1,\Delta_2,\Delta_3,\Delta_4).
\]
The link $\operatorname{Lk}(e)$ is a cycle graph $C_6$; we denote its $1$-cells (edges of the
link) by
\[
\{ a_1,\dots,a_6\},
\]
labelled in counterclockwise order.

We choose orientations of all $2$-cells so that the boundary coefficients on the row of the
fixed edge $e$ are all $+1$. This is always possible because we are free to reverse the orientation
of each $2$-cell, independently.

\subsection{Phase assignments and cocycle condition}

To each edge $f\in E$ we assign a phase $\phi_f \in \mathbb R/2\pi\mathbb Z$. This gives
a group homomorphism
\[
\Phi : C_1 \to \mathbb R/2\pi\mathbb Z,
\]
defined by extending $\Phi(e) = \phi_e$ linearly over $\mathbb Z$.

\begin{definition}[Phase cocycle condition]
We say that $\Phi$ satisfies the phase cocycle condition if for every $2$-cell $s\in F$ we have
\[
\Phi(\partial_2 s) = 0 \in \mathbb R/2\pi\mathbb Z.
\]
Equivalently,
\[
\Phi(\operatorname{im}\partial_2) = 0.
\]
\end{definition}

The physical interpretation is that the total phase around any $2$-cell (face) boundary is
a multiple of $2\pi$, hence trivial in $\mathbb R/2\pi\mathbb Z$.

\newpage

\section{Quotient Group and Induced Homomorphism}

\subsection{The quotient group \(A = C_1 / \operatorname{im}\partial_2\)}

Define the abelian group
\[
A := C_1 / \operatorname{im}\partial_2.
\]
An element of $A$ is the class $[c]$ of a $1$--chain $c\in C_1$, with
\[
[c] = [c'] \iff c-c' \in \operatorname{im}\partial_2.
\]

\begin{proposition}[Induced homomorphism]\label{prop:inducedPhi}
Assume $\Phi(\operatorname{im}\partial_2)=0$. Then $\Phi$ descends to a well-defined
group homomorphism
\[
\overline\Phi : A \longrightarrow \mathbb R/2\pi\mathbb Z,\qquad
\overline\Phi([c]) := \Phi(c).
\]
\end{proposition}

\begin{proof}
Suppose $[c] = [c']$ in $A$. Then $c-c' = \partial_2 Z$ for some $Z\in C_2$.
By hypothesis,
\[
\Phi(c) - \Phi(c') = \Phi(c-c') = \Phi(\partial_2 Z) = 0,
\]
so $\Phi(c) = \Phi(c')$ in $\mathbb R/2\pi\mathbb Z$. Thus $\overline\Phi$ is well defined.
It is clearly a group homomorphism because $\Phi$ is.
\end{proof}

\subsection{Quantization from torsion classes}

Let $[e] \in A$ be the class of a fixed edge $e\in E$.

\begin{proposition}[Quantization from finite order]\label{prop:torsion-quant}
Let $[e]\in A$ be of finite order
\[
k := \operatorname{ord}([e]) < \infty.
\]
Then the phase $\phi_e = \Phi(e)$ satisfies
\[
k\phi_e = 0 \quad (\bmod 2\pi),
\]
hence
\[
\phi_e \in \frac{2\pi}{k}\mathbb Z \quad (\bmod 2\pi).
\]
\end{proposition}

\begin{proof}
By definition of the order, $k[e]=0$ means that $ke\in\operatorname{im}\partial_2$.
Thus there exists $Z\in C_2$ such that
\[
\partial_2 Z = ke.
\]
Applying $\Phi$ and using $\Phi(\operatorname{im}\partial_2)=0$ gives
\[
k\phi_e = \Phi(ke) = \Phi(\partial_2 Z) = 0 \quad (\bmod 2\pi).
\]
Rewriting,
\[
\phi_e \in \frac{2\pi}{k}\mathbb Z \quad (\bmod 2\pi).
\]
\end{proof}

In particular, a nontrivial torsion class $[e]$ forces the phase along $e$ to be discretized
in rational multiples of $2\pi$.

\newpage

\section{Local Structure Around an Edge in the FCC Complex}

We now focus on a single edge $e\in E$ and analyze its local configuration in the FCC complex.
This will allow us to compute the order of $[e]\in A$.

\subsection{Local star and link}

Recall that the local star $\operatorname{St}(e)$ consists of six $2$-cells:
\[
(Q_1,Q_2,\Delta_1,\Delta_2,\Delta_3,\Delta_4),
\]
where $Q_1,Q_2$ are quadrilaterals (coming from octahedra) and each $\Delta_i$ is a triangle
(from tetrahedra).

The link $\operatorname{Lk}(e)$ is combinatorially a hexagon $C_6$, with oriented edges
$a_1,\dots,a_6$ forming a cycle. The orientations of $Q_i,\Delta_j$ are chosen so that
the coefficient of $e$ in $\partial_2 Q_i$ and $\partial_2 \Delta_j$ is \(+1\) for each $i,j$.

\subsection{Relevant block of the boundary matrix}

Consider the restriction of $\partial_2 : C_2 \to C_1$ to the span of the six faces
$(Q_1,Q_2,\Delta_1,\Delta_2,\Delta_3,\Delta_4)$, and to the span of the seven edges
\[
\{e, a_1,\dots,a_6\},
\]
where $a_i$ are the link edges in $\operatorname{Lk}(e) \cong C_6$.

Label the six faces in some fixed order and write a $2$--chain
\[
Z = (z_1,\dots,z_6)^T \in \mathbb Z^6
\]
corresponding to integer coefficients of these six $2$-cells. In this basis, the relevant
$7\times 6$ block of the boundary matrix representing $\partial_2$ has the form:
\begin{equation}\label{eq:local-boundary-matrix}
\partial_2 =
\begin{pmatrix}
1 & 1 & 1 & 1 & 1 & 1 \\
-1 & 1 & 0 & 0 & 0 & 0\\
0 & -1 & 1 & 0 & 0 & 0\\
0 & 0 & -1 & 1 & 0 & 0\\
0 & 0 & 0 & -1 & 1 & 0\\
0 & 0 & 0 & 0 & -1 & 1\\
1 & 0 & 0 & 0 & 0 & -1
\end{pmatrix}.
\end{equation}
Here:
\begin{itemize}[nosep]
  \item the first row corresponds to the coefficient of the edge $e$ in $\partial_2 Z$,
  \item the subsequent six rows correspond to the coefficients of the link edges
        $a_1,\dots,a_6$ in $\partial_2 Z$, capturing a cyclic difference pattern.
\end{itemize}

\newpage

\section{Upper Bound: \texorpdfstring{$\operatorname{ord}([e])\mid 12$}{ord([e]) divides 12}}

We now show that $12e$ is a boundary, hence the order of $[e]$ divides $12$.

\begin{lemma}[Local integer solution for $12e$]\label{lem:12e-boundary}
There exists a $2$--chain $Z\in C_2$ supported in $\operatorname{St}(e)$ such that
\[
\partial_2 Z = 12e.
\]
In particular, $12e \in \operatorname{im}\partial_2$ and thus $12[e]=0$ in $A$.
\end{lemma}

\begin{proof}
In terms of the local coordinates $Z = (z_1,\dots,z_6)^T\in\mathbb Z^6$, the equation
$\partial_2 Z = 12e$ in the $7\times 6$ block \eqref{eq:local-boundary-matrix} becomes
\[
\partial_2 Z =
\begin{pmatrix}
12 \\ 0 \\ \vdots \\ 0
\end{pmatrix}.
\]
Explicitly, this is the system
\begin{align*}
\text{(link edges)}\quad & -z_1 + z_2 = 0,\\
& -z_2 + z_3 = 0,\\
& -z_3 + z_4 = 0,\\
& -z_4 + z_5 = 0,\\
& -z_5 + z_6 = 0,\\
& z_1 - z_6 = 0,\\[0.5em]
\text{(edge $e$)}\quad & z_1 + z_2 + z_3 + z_4 + z_5 + z_6 = 12.
\end{align*}
From the first six (link) equations we obtain
\[
z_1 = z_2 = z_3 = z_4 = z_5 = z_6 =: t.
\]
Substituting into the $e$--row equation yields
\[
6t = 12 \quad\Rightarrow\quad t = 2.
\]
Thus
\[
Z = (2,2,2,2,2,2)^T
\]
is an integer solution and we indeed have
\[
\partial_2 Z =
\begin{pmatrix}
1 & 1 & 1 & 1 & 1 & 1 \\
\vdots & & & & & \vdots
\end{pmatrix}
\begin{pmatrix}
2 \\ 2 \\ 2 \\ 2 \\ 2 \\ 2
\end{pmatrix}
=
\begin{pmatrix}
12 \\ 0 \\ \vdots \\ 0
\end{pmatrix}
= 12e
\]
in the local basis.

Hence $12e \in \operatorname{im}\partial_2$, so $12[e]=0$ in $A$.
\end{proof}

\begin{corollary}[Upper bound on the order]\label{cor:upper-bound}
The order of $[e]\in A$ divides $12$:
\[
\operatorname{ord}([e]) \mid 12.
\]
\end{corollary}

\begin{proof}
By Lemma~\ref{lem:12e-boundary}, we have $12[e]=0$. By definition, the order of $[e]$
is the minimal positive integer $k$ such that $k[e]=0$. Hence $k$ must divide $12$.
\end{proof}

This proves the upper bound on $\operatorname{ord}([e])$.

\newpage

\section{Lower Bound via Phase Cocycles and LCM of Loop Lengths}

We now show that no nonzero multiple of $[e]$ with coefficient less than $12$ can vanish.
Equivalently, we prove
\[
12 \mid \operatorname{ord}([e]),
\]
which, combined with Corollary~\ref{cor:upper-bound}, will yield $\operatorname{ord}([e])=12$.

\subsection{Phase cocycles and induced map on $A$}

Assume $\Phi : C_1 \to \mathbb R/2\pi\mathbb Z$ satisfies the phase cocycle condition,
\[
\Phi(\operatorname{im}\partial_2)=0.
\]
By Proposition~\ref{prop:inducedPhi}, this induces a group homomorphism
\[
\overline\Phi : A \to \mathbb R/2\pi\mathbb Z,\qquad \overline\Phi([c]) = \Phi(c).
\]

We will construct phase assignments so that $\overline\Phi([e])$ has exact order $12$ in
$\mathbb R/2\pi\mathbb Z$. The general group-theoretic fact used is:

\begin{lemma}[Order under homomorphisms]\label{lem:order-divides}
Let $f:A\to B$ be a group homomorphism of abelian groups and $a\in A$ an element of finite order.
Then
\[
\operatorname{ord}(f(a)) \mid \operatorname{ord}(a).
\]
\end{lemma}

\begin{proof}
Let $n = \operatorname{ord}(a) < \infty$, so $na=0$ and $ka\neq0$ for any $0<k<n$.
Then
\[
f(na) = nf(a) = 0_B.
\]
Hence $\operatorname{ord}(f(a))$ divides $n$.
\end{proof}

Thus, if we can realize $\overline\Phi([e])$ as an element of order $12$ in
$\mathbb R/2\pi\mathbb Z$, we obtain $12 \mid \operatorname{ord}([e])$.

\subsection{Shared quadrilateral--triangle edge and a symmetric phase Ansatz}

Consider an edge $e$ which is simultaneously a side of a quadrilateral face and a triangular face.
This is always the case in the FCC complex. Let $\Box$ be a quadrilateral loop
with boundary edges $(e_1,e_2,e_3,e_4)$, and let $\triangle$ be a triangular loop
with boundary edges $(f_1,f_2,f_3)$, so that
\[
e_1 = f_1 = e
\]
is the shared edge.

We lift phases from $\mathbb R/2\pi\mathbb Z$ to $\mathbb R$ by choosing representatives
$\widetilde\phi_f\in\mathbb R$ such that
\[
\widetilde\phi_f \equiv \phi_f \pmod{2\pi}.
\]
We look for a highly symmetric phase assignment in which all boundary edges of
$\Box$ and $\triangle$ have the same lifted phase $\theta$:
\[
\widetilde\phi_{e_i} = \widetilde\phi_{f_j} = \theta \qquad (\forall i,j).
\]
The phase cocycle condition on these two faces then reads:
\begin{align*}
\text{Quadrilateral:}\quad
&\widetilde\phi_{e_1} + \widetilde\phi_{e_2} + \widetilde\phi_{e_3} + \widetilde\phi_{e_4}
= 4\theta = 2\pi n_{\Box},\quad n_{\Box}\in\mathbb Z,\\
\text{Triangle:}\quad
&\widetilde\phi_{f_1} + \widetilde\phi_{f_2} + \widetilde\phi_{f_3}
= 3\theta = 2\pi n_\triangle,\quad n_\triangle\in\mathbb Z.
\end{align*}

Let
\[
x := \frac{\theta}{2\pi}.
\]
Then the above conditions become
\[
4x = n_{\Box}\in\mathbb Z,\qquad 3x = n_\triangle\in\mathbb Z.
\]
Hence
\[
x \in \frac{1}{4}\mathbb Z \cap \frac{1}{3}\mathbb Z
= \frac{1}{\operatorname{lcm}(4,3)}\mathbb Z
= \frac{1}{12}\mathbb Z.
\]
Thus there exists an integer $k$ such that
\[
x = \frac{k}{12},\qquad \theta = 2\pi x = \frac{2\pi k}{12}.
\]

In particular, choosing $k=1$ we obtain a legitimate phase assignment on these loops for which
\[
\phi_e \equiv \theta \equiv \frac{2\pi}{12} \quad (\bmod 2\pi)
\]
on the shared edge $e$.

\begin{remark}
The above symmetric Ansatz concerns only the local loops $\Box$ and $\triangle$.
To obtain a global phase cocycle $\Phi$ on $C_1$ satisfying $\Phi(\partial_2 s)=0$ for
\emph{all} faces $s\in F$, one must check global consistency of these local constraints.
In what follows, we assume that the FCC complex is taken on a periodic domain (e.g.\ a
$3$--torus fundamental domain), and that a global solution $\Phi$ exists extending
this local pattern; equivalently, we assume the existence of a phase cocycle whose
restriction to the neighborhood of $e$ has the above symmetric form.
\end{remark}

Under this assumption, we obtain:

\begin{lemma}[Existence of an order-$12$ phase value]\label{lem:order12-phase}
There exists a phase cocycle $\Phi : C_1\to\mathbb R/2\pi\mathbb Z$ such that
\[
\overline\Phi([e]) = \Phi(e) = \frac{2\pi}{12}.
\]
In particular, the element $\overline\Phi([e])$ has order $12$ in $\mathbb R/2\pi\mathbb Z$:
\[
12\cdot \frac{2\pi}{12} = 2\pi \equiv 0,\quad
m\cdot \frac{2\pi}{12} \equiv 0 \pmod{2\pi} \Rightarrow 12\mid m.
\]
\end{lemma}

\begin{proof}
By the symmetric Ansatz, we can assign $\phi_e \equiv \frac{2\pi}{12}$ to the shared edge
$e$, and choose phases on other edges so that all $2$-cell boundary sums are multiples
of $2\pi$ and the assignment extends to a global phase cocycle $\Phi$.
Then
\[
\overline\Phi([e]) = \Phi(e) = \frac{2\pi}{12}.
\]
This element has order $12$ in $\mathbb R/2\pi\mathbb Z$ by direct inspection:
\[
12\cdot\frac{2\pi}{12} = 2\pi \equiv 0
\]
and if $m\cdot\frac{2\pi}{12} \equiv 0\ (\bmod 2\pi)$, then $2\pi \mid m\cdot\frac{2\pi}{12}$
so $12\mid m$.
\end{proof}

\subsection{Lower bound on \texorpdfstring{$\operatorname{ord}([e])$}{ord([e])}}

\begin{proposition}[Lower bound]\label{prop:lower-bound}
Under the above assumptions on the existence of a phase cocycle $\Phi$ with
$\Phi(e)=2\pi/12$, we have
\[
12 \mid \operatorname{ord}([e]).
\]
\end{proposition}

\begin{proof}
By Lemma~\ref{lem:order12-phase}, there exists a phase cocycle $\Phi$ with
\[
\overline\Phi([e]) = \frac{2\pi}{12},
\]
so $\overline\Phi([e])$ has order $12$ in $\mathbb R/2\pi\mathbb Z$.
By Lemma~\ref{lem:order-divides} applied to $f = \overline\Phi$ and $a = [e]$,
\[
\operatorname{ord}(\overline\Phi([e])) \mid \operatorname{ord}([e]).
\]
Since the left-hand side is $12$, this implies
\[
12 \mid \operatorname{ord}([e]).
\]
\end{proof}

\newpage

\section{Main Result: Exact Order and Quantization Condition}

Combining the upper and lower bounds we obtain the main structural result.

\begin{theorem}[Exact order of the edge class]\label{thm:ord-e-12}
For the fixed edge $e\in E$ in the FCC nearest-neighbour complex, we have
\[
\operatorname{ord}([e]) = 12
\]
in the quotient group $A = C_1/\operatorname{im}\partial_2$.
\end{theorem}

\begin{proof}
By Corollary~\ref{cor:upper-bound}, $\operatorname{ord}([e])$ divides $12$.
By Proposition~\ref{prop:lower-bound}, $12$ divides $\operatorname{ord}([e])$.
Thus by basic number theory,
\[
\operatorname{ord}([e]) = 12.
\]
\end{proof}

As a direct corollary, we obtain the desired divisibility condition on the coefficient
of $e$ in any boundary.

\begin{corollary}[Divisibility of edge coefficient]\label{cor:divisibility-k}
Let $Z\in C_2$ be any $2$--chain such that
\[
\partial_2 Z = k e
\]
for some integer $k\in\mathbb Z$. Then
\[
12 \mid k.
\]
\end{corollary}

\begin{proof}
In $A$, we have
\[
0 = [\partial_2 Z] = k[e].
\]
Since $\operatorname{ord}([e])=12$, by definition we have
\[
k[e]=0 \quad\Longleftrightarrow\quad 12\mid k.
\]
Thus $12$ divides $k$.
\end{proof}

Finally, we translate this back into a quantization statement for the phase along $e$.

\begin{corollary}[Quantization of the edge phase]\label{cor:phase-quantization}
Let $\Phi : C_1 \to \mathbb R/2\pi\mathbb Z$ be a phase cocycle and let $\phi_e = \Phi(e)$.
Then
\[
\phi_e \in \frac{2\pi}{\operatorname{ord}([e])}\mathbb Z \quad (\bmod 2\pi)
= \frac{2\pi}{12}\mathbb Z \quad (\bmod 2\pi).
\]
\end{corollary}

\begin{proof}
Immediate from Proposition~\ref{prop:torsion-quant} and Theorem~\ref{thm:ord-e-12}.
\end{proof}

\newpage

\section*{Conclusion}

We have described a local computation in the FCC nearest-neighbour complex which shows that
the edge class $[e]\in C_1/\operatorname{im}\partial_2$ has exact order $12$. The argument
uses only the local boundary relations around $e$ and the existence of phase cocycles whose
face boundary sums vanish modulo $2\pi$. As a consequence, any $2$--chain whose boundary
is a multiple of a single edge $e$ must have that multiple divisible by $12$, and the phase
assigned to $e$ is correspondingly quantized in units of $2\pi/12$.

\end{document}

