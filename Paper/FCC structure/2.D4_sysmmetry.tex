\documentclass[12pt]{article}
\usepackage{amsmath, amssymb, amsthm}
\usepackage{geometry}
\usepackage{graphicx}
\usepackage{booktabs}
\usepackage{hyperref}
\geometry{a4paper, margin=1in}
\setlength{\parskip}{0.7em}
\setlength{\parindent}{0pt}

\title{\textbf{Counting Distinct Plaquette Phase Configurations under Dihedral Symmetry}}
\author{Francis W. Han \\ \small Ylem Invest}
\date{\today}

\begin{document}

\maketitle

\begin{abstract}
We consider a square plaquette whose four directed links carry distinct quantized phases
\((a,b,c,d)\).
When rotations and reflections of the plaquette are regarded as symmetry operations,
two assignments related by such a symmetry are considered equivalent.
By applying Burnside’s lemma to the dihedral group \(D_4\),
we compute the number of inequivalent configurations.
The result shows that there exist precisely three distinct configurations,
consistent with the classical formula \((n-1)!/2\) for \(n=4\).
This simple enumeration highlights the combinatorial structure
of discrete gauge-phase arrangements on a symmetric lattice plaquette.
\end{abstract}

\section{Introduction}

In lattice models of gauge theory and condensed matter systems,
a plaquette often serves as the minimal closed loop through which
link phases or fluxes are defined~\cite{harary1969graph}.
When link variables are quantized, their relative order around the plaquette
becomes physically meaningful only up to the symmetries of the square.
Counting such equivalence classes of assignments reveals
how many genuinely distinct local configurations exist under lattice symmetry.

This paper derives that count for the simplest case:
a plaquette with four distinct link phases \((a,b,c,d)\)
and the full dihedral symmetry group \(D_4\) acting on it.
Despite the simplicity, the problem directly parallels
the classification of discrete flux patterns or phase windings
under rotational and mirror symmetries~\cite{conway1976numbers}.

\section{Mathematical Setting}

Let
\[
X = \{\, (x_1,x_2,x_3,x_4) \mid (x_1,x_2,x_3,x_4)
\ \text{is a permutation of}\ (a,b,c,d)\,\},
\]
so that \(|X| = 4! = 24.\)

The dihedral group of the square,
\[
D_4 = \{ e, r_{90}, r_{180}, r_{270}, s_x, s_y, s_d, s_{d'} \},
\]
acts on \(X\) by permuting indices according to the square’s rotations
and reflections~\cite{burnside1897theory}.
Two configurations are \emph{equivalent} if they lie in the same orbit
of this action.

The goal is to determine the number of distinct orbits \(|X/D_4|\).

\section{Burnside’s Lemma}

For a finite group \(G\) acting on a finite set \(X\),
\begin{equation}
|X/G| = \frac{1}{|G|} \sum_{g\in G} \mathrm{Fix}(g),
\label{eq:burnside}
\end{equation}
where \(\mathrm{Fix}(g)\) denotes the number of elements of \(X\)
fixed by the group element \(g\)~\cite{polya1987combinatorial}.

Here, \(|G| = |D_4| = 8\) and \(|X| = 24.\)

\section{Fixed Configurations}

\begin{center}
\begin{tabular}{lll}
\toprule
Element \(g\) & Description of action & \(\mathrm{Fix}(g)\) \\
\midrule
\(e\) & Identity & 24 \\
\(r_{90}, r_{270}\) & 4-cycle permutations & 0 \\
\(r_{180}\) & Swaps opposite links \((1\,3)(2\,4)\) & 0 \\
\(s_x,s_y,s_d,s_{d'}\) & Reflections swapping link pairs & 0 \\
\bottomrule
\end{tabular}
\end{center}

Since all phases are distinct, no nontrivial symmetry can fix a configuration.
Substituting into~\eqref{eq:burnside} gives
\[
|X/D_4| = \frac{1}{8}(24) = 3.
\]

\section{Result and Interpretation}

\newtheorem{theorem}{Theorem}
\begin{theorem}
Let four distinct phases \((a,b,c,d)\) be assigned to the four directed links
of a square plaquette, and identify configurations related by the dihedral group \(D_4\).
Then the number of inequivalent configurations is
\[
N = 3.
\]
\end{theorem}

This coincides with the classical combinatorial expression
for cyclic arrangements of \(n\) distinct symbols under reflection:
\[
N = \frac{(n-1)!}{2}, \qquad (n=4).
\]

Physically, the three equivalence classes correspond to
the three distinct cyclic orderings in which
four distinct quantized link phases can circulate around the plaquette,
consistent with rotational and mirror symmetry.

\section{Discussion}

This enumeration illustrates how symmetry drastically reduces
the apparent combinatorial complexity of local gauge configurations.
Although \(24\) raw permutations exist,
the square’s eight symmetry operations identify them into only three types.
Such reductions underpin the combinatorial basis of
symmetry-protected flux quantization and discrete gauge structures.

In general, for \(n\) distinct directed link variables
on an \(n\)-gon with dihedral symmetry \(D_n\),
the number of inequivalent configurations follows the same pattern:
\[
N(n) = \frac{(n-1)!}{2},
\]
a result that smoothly generalizes the current derivation
and mirrors P\'olya’s enumeration framework~\cite{polya1987combinatorial}.

\section{Conclusion}

Applying Burnside’s lemma to the dihedral group \(D_4\),
we established that four distinct link phases on a square plaquette
produce exactly three inequivalent configurations under full symmetry.
This provides a compact combinatorial foundation
for classifying local phase arrangements in discrete lattice models
and related gauge systems.

\section*{References}

\begin{thebibliography}{9}
\bibitem{polya1987combinatorial}
G.~P\'olya and R.~C.~Read,
\emph{Combinatorial Enumeration of Groups, Graphs, and Chemical Compounds}.
Springer, 1987.

\bibitem{conway1976numbers}
J.~H.~Conway, R.~K.~Guy, et~al.,
\emph{On Numbers and Games}.
Academic Press, 1976.

\bibitem{harary1969graph}
F.~Harary,
\emph{Graph Theory}.
Addison--Wesley, 1969.

\bibitem{burnside1897theory}
W.~Burnside,
\emph{Theory of Groups of Finite Order}.
Cambridge University Press, 1897.
\end{thebibliography}

\end{document}
