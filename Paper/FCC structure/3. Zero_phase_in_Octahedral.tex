\documentclass[11pt]{article}
\usepackage{amsmath, amssymb, amsthm, mathtools, geometry, enumitem}
\geometry{margin=1in}

\title{From Dihedral Orbits of Quantized Plaquette Phases to Octahedral Realizability: \\
A Complete Classification in the \(\pi/6\) Lattice Setting}
\author{Francis W. Han}
\date{\today}

\begin{document}
\maketitle

\begin{abstract}
We classify all admissible four-tuples of link phases on a square plaquette when
phases are quantized in units of \(\pi/6\) and restricted to \((-\pi,\pi]\), subject to the
plaquette closure condition. We then determine exactly which of these can be assigned,
up to cyclic and reflective symmetries, to the three orthogonal square loops (in the
\(XY\), \(YZ\), and \(ZX\) planes) of an octahedron so that all eight triangular faces have
zero phase sum modulo \(2\pi\). The pipeline is:
\[
\text{(I) 42 plaquette quadruples}\ \longrightarrow\
\text{(II) realizability theorem}\ \longrightarrow\ 
\]
\[ \text{(III) operational subset (11)}\ \longrightarrow\
\text{(IV) 9 final classes (sign pairing).}
\]
The \(\pi/6\) quantization follows from a \(\mathbb{Z}/12\mathbb{Z}\) torsion in the FCC 2--complex,
while dihedral (\(D_4\)) symmetry controls plaquette equivalence under cyclic and mirror
actions.
\end{abstract}

\section{Setting and notation}

All phases are expressed as integer multiples of \(\pi/6\). Write
\[
(a,b,c,d) \;=\; \tfrac{\pi}{6}(k_1,k_2,k_3,k_4),\qquad
k_i \in \{-5,-4,\dots,6\},\quad k_1<k_2<k_3<k_4,
\]
and impose the plaquette closure
\[
k_1+k_2+k_3+k_4 \equiv 0 \pmod{12}.
\]
The ordering \(k_1<k_2<k_3<k_4\) is just a canonical representative choice; cyclic/reflective
reorderings of the same four values are regarded as dihedral \(D_4\)-equivalent when we
consider a plaquette as a square.

\paragraph{Background and context.}
The \(\pi/6\) quantization can be derived directly from the geometry/topology of the FCC
2--complex: the link class has the order \(12\) in the relevant homology/cokernel, hence
\(\phi_e \in (\pi/6)\mathbb{Z} \pmod{2\pi}\) [1]. The dihedral reduction on a square
plaquette is a standard Burnside/Polya exercise [2].

\section{Stage I: the 42 admissible plaquette quadruples}

Direct enumeration under the rules above yields precisely \(42\) admissible ordered sets
\((k_1,k_2,k_3,k_4)\). Grouped by integer sum \(S:=k_1+k_2+k_3+k_4\in\{-12,0,12\}\), they are:

\subsection*{Sum \(-12\) (2 cases)}
\[
(-5,-4,-3,0),\qquad (-5,-4,-2,-1).
\]

\subsection*{Sum \(0\) (31 cases)}
\begin{align*}
&(-5,-4,3,6),\,(-5,-4,4,5);\\
&(-5,-3,2,6),\,(-5,-3,3,5);\\
&(-5,-2,1,6),\,(-5,-2,2,5),\,(-5,-2,3,4);\\
&(-5,-1,0,6),\,(-5,-1,1,5),\,(-5,-1,2,4);\\
&(-5,0,1,4),\,(-5,0,2,3);\\
&(-4,-3,1,6),\,(-4,-3,2,5),\,(-4,-3,3,4);\\
&(-4,-2,0,6),\,(-4,-2,1,5),\,(-4,-2,2,4);\\
&(-4,-1,0,5),\,(-4,-1,1,4),\,(-4,-1,2,3);\\
&(-4,0,1,3);\\
&(-3,-2,-1,6),\,(-3,-2,0,5),\,(-3,-2,1,4),\,(-3,-2,2,3);\\
&(-3,-1,0,4),\,(-3,-1,1,3);\\
&(-3,0,1,2);\\
&(-2,-1,0,3),\,(-2,-1,1,2).
\end{align*}

\subsection*{Sum \(12\) (9 cases)}
\[
(-3,4,5,6),\quad (-2,3,5,6),\quad (-1,2,5,6),\,(-1,3,4,6),
\]
\[
(0,1,5,6),\,(0,2,4,6),\,(0,3,4,5),\,(1,2,3,6),\,(1,2,4,5).
\]

\noindent
These \(42\) are ``raw'' canonical representatives; any cyclic/reflective rearrangement of
a given quadruple describes the same plaquette up to \(D_4\).

\section{Stage II: octahedral realizability}

Place three square plaquettes in the \(XY\), \(YZ\), and \(ZX\) planes of a unit octahedron
with vertices \((\pm1,0,0)\), \((0,\pm1,0)\), \((0,0,\pm1)\). The three plaquettes carry the
\emph{same} four values as above, but each plane may take a cyclic/reflective permutation
of the 4--set. For the resulting octahedron, each of the eight triangular faces must have
phase sum \(0\pmod{2\pi}\), i.e. in integer units \(0\pmod{12}\).

Let \(K=\{k_1,k_2,k_3,k_4\}\subset\{-5,\dots,6\}\) with \(k_1<k_2<k_3<k_4\) and
\(\sum K\equiv0\pmod{12}\). There exists an assignment of \(K\) to the three orthogonal
plaquettes (each as a cyclic/reflective permutation of \(K\)) such that all eight triangular
faces of the octahedron have phase sum \(0\pmod{12}\) \emph{if and only if} \(0\in K\).


\begin{proof}[Sketch (sufficiency)]
Write \(K=\{0,x,y,z\}\) with \(x<y<z\) and \(x{+}y{+}z\equiv0\pmod{12}\). Assign
\[
\mathrm{XY}=[0,x,y,z],\quad
\mathrm{YZ}=[0,y,z,x],\quad
\mathrm{ZX}=[0,z,x,y].
\]
Column sums over the three planes are \((0,\,x{+}y{+}z,\,x{+}y{+}z,\,x{+}y{+}z)\equiv(0,0,0,0)\),
so every triangular face closes.
\end{proof}

\begin{proof}[Sketch (necessity)]
If \(0\notin K\), then analyzing the allowed column sums \(T_j\in\{-12,0,12\}\) and the way
\(\pm6\) can appear in a column (it forces the two companions to be one of
\((-5,-1),(-4,-2),(1,5),(2,4)\)) shows there is no way---using only permutations of the
\emph{same} 4--set in three rows---to produce four columns all summing to \(0\pmod{12}\)
simultaneously; one column inevitably lands in a nonzero residue class. (This is a short
double-counting/compatibility obstruction once one accounts for the fact each element of
\(K\) must be used exactly once per row.) Hence realizability fails.
\end{proof}

\paragraph{Output of Stage II.}
Exactly those admissible quadruples that \emph{contain \(0\)} are realizable on the octahedron.
From the list of 42, this leaves the following 14:
\[
\begin{array}{l}
\{-5,-4,-3,0\},\ \{-5,-1,0,6\},\ \{-5,0,1,4\},\ \{-5,0,2,3\},\\[2pt]
\{-4,-2,0,6\},\ \{-4,-1,0,5\},\ \{-4,0,1,3\},\\[2pt]
\{-3,-2,0,5\},\ \{-3,-1,0,4\},\ \{-3,0,1,2\},\ \{-2,-1,0,3\},\\[2pt]
\{0,1,5,6\},\ \{0,2,4,6\},\ \{0,3,4,5\}.
\end{array}
\]

\section{Stage III: an operational subset (11)}
In many applications it is convenient to discard the three \emph{positively} summed cases
with \(\sum K=12\), keeping only nonpositive representatives. Removing
\(\{0,1,5,6\}\), \(\{0,2,4,6\}\), \(\{0,3,4,5\}\) from the 14 realizable sets leaves an
``operational'' subset of 11:
\[
\begin{array}{l}
\{-5,-4,-3,0\},\ \{-5,-1,0,6\},\ \{-5,0,1,4\},\ \{-5,0,2,3\},\\[2pt]
\{-4,-2,0,6\},\ \{-4,-1,0,5\},\ \{-4,0,1,3\},\\[2pt]
\{-3,-2,0,5\},\ \{-3,-1,0,4\},\ \{-3,0,1,2\},\ \{-2,-1,0,3\}.
\end{array}
\]
(One can adopt other conventions; this choice simply removes the three \(+\!12\) cases and
keeps a single representative whenever a sign partner exists within range.)

\section{Stage IV: final 9 classes via sign pairing}
Finally, identify sets that differ only by a global sign, whenever the sign-flipped set also
lies in the allowed range \(\{-5,\dots,6\}\). Writing \(K\sim -K\), the realizable families collapse
to the following 9 equivalence classes:
\begin{itemize}[leftmargin=2em]
  \item \(\{-5,-4,-3,0\}\ \sim\ \{0,3,4,5\}\) \hfill (pair)
  \item \(\{-5,0,1,4\}\ \sim\ \{-4,-1,0,5\}\) \hfill (pair)
  \item \(\{-5,0,2,3\}\ \sim\ \{-3,-2,0,5\}\) \hfill (pair)
  \item \(\{-4,0,1,3\}\ \sim\ \{-3,-1,0,4\}\) \hfill (pair)
  \item \(\{-3,0,1,2\}\ \sim\ \{-2,-1,0,3\}\) \hfill (pair)
  \item \(\{-5,-1,0,6\}\) \hfill (singleton; sign-flip involves \(-6\))
  \item \(\{-4,-2,0,6\}\) \hfill (singleton; sign-flip involves \(-6\))
  \item \(\{0,1,5,6\}\) \hfill (singleton; sign-flip involves \(-6\))
  \item \(\{0,2,4,6\}\) \hfill (singleton; sign-flip involves \(-6\))
\end{itemize}
These 9 are the terminal representatives modulo: (i) dihedral permutations on each square,
(ii) the octahedral realizability constraint, and (iii) global sign reversal whenever valid.

\section*{Appendix A: A compact construction for the octahedron}
Given any \(K=\{0,x,y,z\}\) with \(x+y+z\equiv0\pmod{12}\) in the allowed range, the cyclic-shift
pattern
\[
\mathrm{XY}=[0,x,y,z],\quad
\mathrm{YZ}=[0,y,z,x],\quad
\mathrm{ZX}=[0,z,x,y]
\]
makes every triangular face sum to \(0\pmod{12}\). Reversing one row produces alternative
column-sum distributions \((n_{-12},n_0,n_{12})\) while keeping all faces closed.

\section*{References}
\begin{enumerate}[label={[\arabic*]}]
  \item F.\ W.\ Han, \emph{Quantization of Link Phase in the FCC Lattice from Pure Geometric Topology}, Oct.\ 2025. (Derives the \(\pi/6\) unit from a \(\mathbb{Z}/12\mathbb{Z}\) torsion in the FCC 2--complex.) 
  \item F.\ W.\ Han, \emph{Counting Distinct Plaquette Phase Configurations under Dihedral Symmetry}, Oct.\ 2025. (Burnside/dihedral reduction for a square plaquette.)
\end{enumerate}

\end{document}
