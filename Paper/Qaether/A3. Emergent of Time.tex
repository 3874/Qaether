\documentclass{article}
\usepackage{amsmath, amssymb, bm}
\usepackage{geometry}
\geometry{margin=1in}

\title{A3. Emergence of Time: Link $\cdot$ Loop Consistency}
\author{Francis Han}
\date{\today}

\begin{document}
\maketitle

\begin{abstract}
We develop a mathematically consistent framework in which time emerges not as
a background parameter but as a gauge--invariant functional of local activity in
an SU(2) lattice gauge geometry. Two complementary layers are introduced:
link--based proper time derived from plaquette--averaged holonomy, and
loop--based proper time depending only on the conjugation--invariant holonomy
angle of closed contours. The Loop--Equivalence Axiom (ELT) is imposed,
requiring that any two simple loops with identical holonomy angle exhibit the
same proper--time, independent of shape or length. All definitions, domain
conditions, and regularity constraints ensuring full mathematical consistency
are stated explicitly. The continuum limit yields a dependence on 
$F_{\mu\nu}F^{\mu\nu}$, and a structural correspondence with Jacobi--type
path--integral lapses is shown. Coordinate time (graph synchronisation) and
physical dilation (gauge--invariant activity) become sharply distinguished,
forming a coherent emergent--time mechanism.
\end{abstract}

\section{Principle: Time as Emergent Activity}

Time is treated as an emergent quantity determined by local activity.  
For any object (cell, link, loop) with activity--rate $\beta\in(-1,1)$,
\[
\gamma^{-1}=\sqrt{1-\beta^2},
\]
and with $\beta=\tanh(\cdots)$ we guarantee $|\beta|<1$, yielding
\[
d\tau = t_p\sqrt{1-\beta^2},\qquad 0<d\tau\le t_p.
\]

\section{Links and Loops: Gauge--Invariant Geometry}

\paragraph{Links.} 
An oriented link $e=(i\to i+\hat\mu)$ carries
\[
U_\mu(i)=\mathbf q_{i+\hat\mu}\mathbf q_i^{-1}\in SU(2),
\qquad 
U_\mu(i)\mapsto g_{i+\hat\mu}U_\mu(i)g_i^{-1}.
\]

\paragraph{Loops.}
For a closed contour $C$,
\[
W(C)=\prod_{e\in C}U_e,
\qquad 
W(C)\mapsto g_{i_0}W(C)g_{i_0}^{-1}.
\]
Define
\[
\Theta_C=\arccos\!\Bigl(\tfrac12\mathrm{Tr}\,W(C)\Bigr)\in[0,\pi].
\]

\paragraph{Plaquettes.}
Minimal loops satisfy
\[
U_\square=W(C_\square),
\qquad \Theta_\square=\Theta_{C_\square}.
\]

\section{Two Levels of Activity: Link vs.\ Loop}

\subsection{Link--based Activity}

Let $\mathcal P(e)$ be the set of plaquettes containing $e$.  
Assume 
\[
\omega_\square>0,\qquad 
\sum_{\square\ni e}\omega_\square>0.
\]
Define
\[
\Theta_e^{\rm(GI)}
=
\left(
\frac{
\sum_{\square\ni e}\omega_\square\,\Theta_\square^2
}{
\sum_{\square\ni e}\omega_\square
}
\right)^{1/2},
\qquad 
\Theta_\square=\arccos(\tfrac12\mathrm{Tr}\,U_\square).
\]
Boundary links with $\mathcal P(e)=\varnothing$ take $\Theta_{e}^{\rm(GI)}=0$.

The activity and proper time are
\[
\beta_e=\tanh\!\Bigl(\kappa_1\tfrac{\Theta_e^{\rm(GI)}}{\pi}\Bigr),
\qquad 
d\tau_e=t_p\sqrt{1-\beta_e^2}.
\]

\subsection{Loop--based Activity}

For a closed loop $C$,
\[
\beta_C=\tanh\!\Bigl(\kappa_{\rm loop}\tfrac{\Theta_C}{\pi}\Bigr),
\qquad \kappa_{\rm loop}=\text{const},
\]
so the loop proper--time is
\[
T_{\rm loop}(C)=t_p\sqrt{1-\beta_C^2}.
\]

\section{Loop--Equivalence Axiom (ELT)}

\[
\boxed{
\Theta_{C_1}=\Theta_{C_2}
\;\Longrightarrow\;
T_{\rm loop}(C_1)=T_{\rm loop}(C_2)
}
\]
Since $T_{\rm loop}$ depends only on $\Theta_C$ and 
$\kappa_{\rm loop}$ is loop--independent, the axiom holds.

\section{Coordinate Time vs.\ Proper Time}

For a worldline $\gamma$,
\[
\tau[\gamma]
=
\sum_{e\in\gamma} t_p\sqrt{1-\beta_e^2},
\qquad 
\beta_e=\tanh\!\Bigl(\kappa_1\tfrac{\Theta_{e}^{\rm(GI)}}{\pi}\Bigr).
\]
This differs from the intrinsic $T_{\rm loop}(C)$ of loop objects.

\section{Cell and Block Effective Proper Time}

\[
\mathcal A_i^2
=
c_F\Bigl(\tfrac{\|\mathcal F_i\|}{\Omega_F}\Bigr)^2
+
c_R\Bigl(\tfrac{\|F_\star(i)\|}{\Omega_R}\Bigr)^2
+
c_\Omega\Bigl(\tfrac{\|\nabla\mathbf q_i\|}{\Omega_\Omega}\Bigr)^2,
\qquad 
\beta_i=\tanh(\kappa\mathcal A_i),
\]
\[
d\tau_i=t_p\sqrt{1-\beta_i^2}.
\]

Define the pressure weight
\[
P_i=p_0(1-\alpha m_i)F_\star(i),
\qquad 
F_\star(i)\ge0,\quad 0\le\alpha m_i<1.
\]
For a block $B$,
\[
\Delta T_{\rm eff}(B)
=
\frac{\sum_{i\in B}P_i\,d\tau_i}{\sum_{i\in B}P_i},
\qquad 
\sum_{i\in B}P_i>0.
\]

\section{Local YM Coupling}

\[
\mathcal L_{\rm YM}(i)
=
\frac{1}{2 g_{\rm eff}^2(i)}\mathrm{Tr}[\mathcal F_{\mu\nu}(i)\mathcal F^{\mu\nu}(i)],
\]
\[
\frac{1}{g_{\rm eff}^2(i)}
=
\frac{1}{g_0^2}
+
\tilde\lambda(1-\alpha m_i)\,F_\star(i).
\]

\section{Branch Continuity}

The SU(2) logarithm uses the principal branch $|\theta|\le\pi$, with nearest--branch 
tracking to avoid $2\pi$ jumps.

\section{Continuum Limit}

\[
U_\square=\exp(ia^2F_{\mu\nu}+O(a^3)),
\qquad 
\Theta_\square\simeq a^2\|F_{\mu\nu}\|.
\]
Thus
\[
T_{\rm loop}(C)
\simeq
t_p\!\left[
1-\tfrac12\left(\kappa_{\rm loop}\tfrac{\Theta_C}{\pi}\right)^2
\right],
\]
consistent with a dependence on $F_{\mu\nu}F^{\mu\nu}$.

\section{Relation to Path--Integral Lapses}

\[
S_J[\gamma]
=
\int d\lambda\,
\sqrt{2(E_{\rm tot}-U_{\rm eff})}\,
\sqrt{\tfrac{ds_{\rm conf}^2}{d\lambda^2}},
\]
\[
d\tau_{\rm eff}
=
\frac{1}{\sqrt2}\,
\frac{\sqrt{ds_{\rm conf}^2}}{\sqrt{E_{\rm tot}-U_{\rm eff}}}.
\]
If $ds_{\rm conf}^2$ and $U_{\rm eff}$ depend only on the angle $\Theta$,
then $d\tau_{\rm eff}=f(\Theta)$, the same class of gauge--invariant scalar
lapse functions as
\[
d\tau=t_p\sqrt{1-\beta^2(\Theta)}.
\]

\section{Summary}

Time along links is synchronised by $t_p$,
while physical dilation is determined by gauge--invariant loop/cell activity.
Loops with identical holonomy angle $\Theta$ share identical proper--time,
independent of shape.

\end{document}
