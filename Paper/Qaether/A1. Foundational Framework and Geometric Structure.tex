\documentclass[12pt]{article}
\usepackage{amsmath, amssymb, amsfonts}
\usepackage{geometry}
\usepackage{graphicx}
\usepackage{booktabs}
\usepackage{array}
\usepackage{physics}
\usepackage{bm}
\geometry{a4paper, margin=1in}

\begin{document}

\title{\textbf{Qaether Theory – Foundational Framework and Geometric Structure}}
\author{Francis W. Han \\ \small Ylem Invest}
\date{\today}
\maketitle

\section*{Introduction: Core Philosophy and Overview}

1. The universe is conceived as a noncommutative quaternionic phase network of discrete minimal spatial units, \textit{Qaether}, each a 3-ball of diameter \( l_p \) arranged in an implicit face-centered cubic (FCC) contact structure. The totality of physical law—particles, fields, and gravitation—emerges solely from the link variables between Qaether nodes and the induced holonomy and curvature thereof.

2. The fundamental entities are defined as follows:
   \begin{itemize}
       \item \textbf{Void}: A pure nothingness containing no variables, metric, or boundary conditions. It provides no coordinates or distances; only the contact relations between Qaether exist within it.
       \item \textbf{Qaether}: The minimal 3-ball \( B^3(l_p/2) \) of diameter \( l_p \), whose internal degree of freedom is a unit quaternion \( \mathbf{q}_i \in SU(2) \cong S^3 \). This encapsulates spin, gauge phase, and topological defect structure. Its intrinsic zero-point energy is defined by \( E_q = \tfrac{1}{2}\hbar\omega_q \), the ground mode of an internal standing wave.
       \item \textbf{Lattice Emergence}: The spatial lattice arises spontaneously through Qaether contact. All physical actions are expressed through the graph of such contacts:
       \[
       V = \{i\}, \quad 
       E = \{(i,j)\,|\,\text{Qaether }i,j\text{ are in contact}\},
       \]
       with link variable \( \Delta \mathbf{q}_{ij} = \mathbf{q}_j \mathbf{q}_i^{-1} \).
   \end{itemize}

\section*{A1. Fundamental Entities: Void and Qaether (\(S^3\))}

\subsection*{1. Void}
Void does not directly participate in dynamics. It acts only as a background of pure nonexistence, supplying implicit adjacency information between Qaether nodes. No metric, coordinates, or distances are defined.

\subsection*{2. Qaether Cell: Physical Volume and Internal Phase Space}

\renewcommand{\arraystretch}{1.2}
\begin{tabular}{p{3cm} p{4cm} p{7cm}}
\toprule
\textbf{Aspect} & \textbf{Mathematical Form} & \textbf{Description} \\
\midrule
Physical Volume & \( B^3(\tfrac{l_p}{2}) \) & 3-ball of diameter \( l_p \), forming the FCC contact lattice. \\
Internal Phase Space & \( S^3 \cong SU(2)_{\text{int}} \) & Each Qaether carries an internal unit quaternion representing unified spin and gauge phase. \\
Quaternion Variable & \( \mathbf{q}_i = n_i^0 + n_i^1\mathbf{i} + n_i^2\mathbf{j} + n_i^3\mathbf{k}, \; \sum (n_i^a)^2 = 1 \) & Can be expressed as rotation angle and axis: \( \mathbf{q}_i = \cos\!\frac{\phi_i}{2}\,\mathbb{I} + i\sin\!\frac{\phi_i}{2}(\mathbf{n}_i\!\cdot\!\boldsymbol{\sigma}) \). \\
\bottomrule
\end{tabular}

\subsection*{3. Standing Wave Mode}
Each cell supports a minimal eigenmode of a scalar or tensor field. Coupling to the quaternionic phase allows spin-\(\tfrac{1}{2}\) excitation. Its eigenfrequency \( \omega_q \) defines the intrinsic zero-point energy \( E_q = \tfrac{1}{2}\hbar\omega_q \).

\section*{A2. Spatial Structure: Graph, Link, and Curvature}

\subsection*{1. Lattice Graph Definition}
\[
V = \{i\,|\,i \text{ is a Qaether index}\}, \quad 
E = \{(i,j)\,|\,\text{Qaether }i,j\text{ are in physical contact}\}.
\]
Two 3-balls of diameter \( l_p \) are in contact when their centers are separated by \( l_p \). The Void offers no coordinate system; only adjacency defines structure.

\subsection*{2. Vertex and Link Variables}

Gauge invariance under local transformations:
\[
\mathbf{q}_i \to g_i\,\mathbf{q}_i, \quad 
\Delta \mathbf{q}_{ij} \to g_j\,\Delta \mathbf{q}_{ij}\,g_i^{-1}.
\]

\begin{tabular}{p{3cm} p{5cm} p{6cm}}
\toprule
\textbf{Item} & \textbf{Definition} & \textbf{Interpretation} \\
\midrule
Vertex Variable & \( \mathbf{q}_i \in SU(2) \) & Internal unit quaternion of each Qaether cell. \\
Link Variable & \( U_{ij} = \mathbf{q}_j\,\mathbf{q}_i^{-1} \) & Relative phase between adjacent cells (noncommutative \( SU(2) \) element). \\
\bottomrule
\end{tabular}

\subsection*{3. Quantization of Phase Difference}
Due to FCC symmetry and energy minimization, total phase difference across each link is quantized as
\[
\Delta\phi_{ij}^{\mathrm{tot}} = n\frac{\pi}{6}, \quad n \in \mathbb{Z},
\]
imposing a residual discrete symmetry \( \mathbb{Z}_{12} \).

\subsection*{4. Plaquette Holonomy and Curvature}

\begin{tabular}{p{4cm} p{10cm}}
\toprule
\textbf{Quantity} & \textbf{Definition} \\
\midrule
Plaquette & Minimal closed loop of four links \( \square \). \\
Holonomy & \( U_{\square} = \prod_{(i,j)\in\square} U_{ij} = \prod_{(i,j)\in\square} e^{i\Delta \phi_{ij}^{\mathrm{tot}}} \). \\
Curvature Scale & \( \Theta_{\square} = \arccos\!\left(\tfrac{1}{2}\operatorname{Tr}U_{\square}\right). \) \\
\bottomrule
\end{tabular}

\end{document}
