\documentclass[12pt]{article}
\usepackage{amsmath, amssymb, amsfonts, bm}
\usepackage{graphicx}
\usepackage{physics}
\usepackage{geometry}
\geometry{margin=1in}

\begin{document}

\section*{A2. Emergence of Mass and Gravity: The Coupled Pressure Model}

\subsection*{1. Cell Surface Variables}
\begin{itemize}
  \item Total reflective boundary area: 
  \[
    \mathfrak{A}_s \approx \pi l_p^2
  \]
  (the external reflective surface of a single Qaether cell).

  \item Area blocked per bond:
  \[
    \mathfrak{A}_b \ll \mathfrak{A}_s 
    \quad \Longrightarrow \quad 
    \alpha \equiv \frac{\mathfrak{A}_b}{\mathfrak{A}_s} \ll 1
  \]
\end{itemize}

\subsection*{2. Remaining Reflective Area}
For a cell \( i \) with \( m_i \) bonds,
\[
\mathfrak{A}_i(m_i) 
= \mathfrak{A}_s - m_i \mathfrak{A}_b 
= (1 - \alpha m_i)\,\mathfrak{A}_s.
\]
Even for the FCC lattice maximum \( m_i = 12 \), we have \( \alpha m_i \ll 1 \), thus \( \mathfrak{A}_i > 0 \).

\subsection*{3. Amplitude Coefficient \( F_\star \): Gauge-Invariant (Plaquette-Based)}
The sum below is taken over all minimal plaquettes sharing site \( i \) (the smallest loops such as squares or triangles on the FCC lattice):
\[
\boxed{
F_\star(i) \equiv 
\frac{
  \sum_{\square \ni i} 
  \omega_\square 
  \bigl(1 - \tfrac{1}{2}\mathrm{Re\,Tr}\,U_\square\bigr)
}{
  \sum_{\square \ni i} 
  \omega_\square
}}.
\]
This definition represents a local average of the standard Wilson density 
\((1 - \tfrac{1}{2}\mathrm{Re\,Tr}\,U_\square)\), hence is gauge invariant.

For small angles,
\[
U_\square \simeq e^{i\Theta_\square},
\qquad 
\tfrac{1}{2}\mathrm{ReTr}\,U_\square \simeq 1 - \tfrac{1}{2}\|\Theta_\square\|^2,
\]
which gives
\[
\boxed{
F_\star(i) \simeq 
\frac{1}{2N_i}\sum_{\square\ni i}\|\Theta_\square\|^2
\;\propto\;
\text{(local curvature intensity average)}.
}
\]

\subsection*{4. Internal Phase Oscillation Energy of Qaether Cell \( i \)}
Assume the Qaether wavelength equals the Planck length \( l_p \).  
The corresponding angular frequency is
\[
\omega_q = \frac{2\pi c}{l_p},
\]
and the internal phase oscillation energy is
\[
E_q = \tfrac{1}{2}\hbar \omega_q = \hbar \frac{\pi c}{l_p}.
\]
Then the phase energy density is
\[
u_\phi = \frac{E_q}{V_s}
= \frac{\tfrac{1}{2}\hbar \omega_q}{\tfrac{1}{6}\pi l_p^3}
= \frac{6\hbar c}{l_p^4}.
\]

\subsection*{5. Reference Pressure \( p_0 \)}
Define the pressure corresponding to 100\% reflection per unit area:
\[
p_0 = 2u_\phi = \frac{12\hbar c}{l_p^4}.
\]
The blocked region (\( m_i \mathfrak{A}_b \)) receives no phase-wave impact, hence zero pressure.

\subsection*{6. Local Effective Pressure}
With effective boundary thickness \( \delta = \eta l_p \) (\( \eta \sim \mathcal{O}(1) \); limit \( \eta \to 0 \) if needed):
\[
\boxed{
P_i(m_i)
= p_0 \frac{\mathfrak{A}_i(m_i)}{\mathfrak{A}_s} F_\star(i)
= p_0 (1 - \alpha m_i) F_\star(i).
}
\]

\subsection*{7. Pressure–Energy Mapping and Mass (Background Difference Definition)}
Stored energy:
\[
U_{\text{press}}(i) = P_i \mathfrak{A}_s \delta 
= p_0 \mathfrak{A}_s \eta l_p (1 - \alpha m_i) F_\star(i).
\]
Relative to the uniform (unobservable) vacuum background:
\[
\boxed{
\Delta U_{\text{press}}(i)
= - \alpha m_i p_0 \mathfrak{A}_s \eta l_p F_\star(i)
= - \alpha m_i \eta 12\pi \frac{\hbar c}{l_p} F_\star(i)
= - \alpha m_i \eta 12\pi E_{\mathrm{Pl}} F_\star(i),
}
\]
where \( E_{\mathrm{Pl}} = \hbar c / l_p \).

Binding energy per bond (uniform per-cell reference):
\[
\boxed{
\Delta U_{\text{bond}}^{(\mathrm{per\ cell})}
= - \alpha \eta 12\pi E_{\mathrm{Pl}} F_\star(i).
}
\]

Thus the effective inertial (rest) mass of a cell:
\[
\boxed{
m_{\mathrm{eff}}(i)
= \frac{E_q + U_{\text{link}}(i) + \Delta U_{\text{press}}(i)}{c^2},
}
\]
where \( U_{\text{link}} \) represents link/plaquette contributions such as \( W_{ij}(\Delta\phi_{ij}) \).
Since \( \Delta U_{\text{press}} < 0 \Rightarrow \) increasing bonds (\( m_i \uparrow \)) reduces rest energy (inertial mass) $\Rightarrow$ stronger binding.

\subsection*{8. Stress–Energy Tensor (Apparent Fluid Approximation: Isotropic + Anisotropic Corrections)}
Isotropic approximation:
\[
T^{00}_i \approx u_\phi (1 - \alpha m_i) F_\star(i), 
\qquad
T^{aa}_i \approx p_0 (1 - \alpha m_i) F_\star(i)
\quad (a = 1,2,3).
\]
Anisotropic correction (bond patch normal \( \hat{\mathbf{n}}_e \)):
\[
\Delta T^{ab}_i \approx - p_0 F_\star(i)
\sum_{e \in \mathcal{N}(i)} \alpha n_e^a n_e^b.
\]
If the bond directions are random/uniform:
\[
\sum_e \alpha n_e^a n_e^b 
\simeq \tfrac{\alpha m_i}{3}\delta^{ab}.
\]

\subsection*{9. Local Moment of Inertia}
Assume the cell’s internal energy is distributed over a sphere of radius \( r = l_p/2 \):
\[
E_q = \frac{\pi \hbar c}{l_p}, 
\qquad
m_q = \frac{E_q}{c^2} = \frac{\pi \hbar}{c l_p}, 
\qquad
r_q = \frac{l_p}{2}.
\]
Then the moment of inertia for a uniform solid sphere:
\[
I_i = \frac{2}{5} m_q r_q^2 
= \frac{2}{5} 
\left(\frac{\pi \hbar}{c l_p}\right)
\left(\frac{l_p}{2}\right)^2.
\]
Simplified:
\[
\boxed{
I_i = \frac{\pi}{10}\frac{\hbar l_p}{c}.
}
\]

\end{document}
