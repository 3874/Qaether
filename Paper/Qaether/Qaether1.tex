\documentclass[12pt,a4paper]{article}
\usepackage{amsmath,amssymb,amsthm,graphicx,bm}
\usepackage{geometry}
\geometry{margin=1in}
\usepackage{physics}
\usepackage{titlesec}
\usepackage{hyperref}
\usepackage{cite}
\usepackage{mathtools}

\titleformat{\section}{\normalfont\large\bfseries}{\thesection.}{0.5em}{}
\titleformat{\subsection}{\normalfont\normalsize\bfseries}{\thesubsection}{0.5em}{}

\title{\textbf{Spin, Electric Charge, and Color Charge on the FCC Lattice: \\ 
A Lattice Realization of the SU(2)--U(1)--SU(3) Gauge Hierarchy}}
\author{Francis W. Han \\ \small Ylem Invest}
\date{\today}

\begin{document}
\maketitle

\begin{abstract}
We present a unified lattice framework in which spin, electric charge, and color charge emerge from the topology of the face--centered cubic (FCC) lattice.
The coexistence of triangular and square minimal loops in the FCC skeleton provides the minimal structure supporting both SO(3) parity and U(1) phase.
A quaternionic SU(2) field at each site encodes spin orientation and local phase; its internal U(1) projection produces quantized electric charge in units of $e/6$.
On the same lattice, combinatorial classes of plaquette phase quadruples under the dihedral group $D_4$ define color charge as SU(3) weight--type invariants.
The $\mathbb{Z}_{12}$ torsion of the FCC 2--skeleton fixes the common quantization scale $\pi/6$, ensuring coherence among spin, charge, and color.
Baryon and meson color neutrality follow naturally from the closure rules of octahedral cells.
\end{abstract}

\section{Geometric Background}

\textbf{FCC Lattice and 2--Skeleton.}
Let $G=(V,E)$ be the nearest--neighbor graph of the face--centered cubic lattice.
We attach triangular and square minimal loops as 2--cells, forming the 2--skeleton
\[
C_2=\mathbb{Z}^F,\qquad C_1=\mathbb{Z}^E,\qquad \partial_2:C_2\to C_1.
\]
For each edge $e\in E$ assign a phase $\phi_e\in\mathbb{R}/2\pi\mathbb{Z}$.
If $\Phi(\partial_2 f)=0$ for all minimal faces $f$, then $\Phi\in Z^1(X;\mathbb{R}/2\pi\mathbb{Z})$.

\textbf{Lattice Torsion.}
From the Smith normal form of the boundary matrix $\partial_2$, one obtains a local torsion element of order $12$ in
\[
A:=C_1/\mathrm{im}\,\partial_2,
\]
implying
\[
\phi_e\in \frac{2\pi}{12}\mathbb{Z}\equiv\frac{\pi}{6}\mathbb{Z}\pmod{2\pi}.
\]
Hence both electric and color charges will share the same quantization scale $\pi/6$.

\section{Spin: Quaternionic SU(2) Field}

At each site $i\in V$, define
\[
q_i=e^{i\phi_i\hat n_i\cdot\vec\sigma}
   =\cos\frac{\phi_i}{2}+i\sin\frac{\phi_i}{2}\,\hat n_i\cdot\vec\sigma
   \in SU(2),
\]
with local axis $\hat n_i$ and phase $\phi_i$.

\subsection{Link and Wilson Loops}

\[
U_{ij}=q_i q_j^{-1}\in SU(2),\qquad
W(\ell)=\prod_{(ij)\in\ell}U_{ij}.
\]
For minimal loops, $W(\ell)\in\{\pm1\}=Z(SU(2))$, recording the even/odd winding of spin-$1/2$.
Mapping to $SO(3)$ identifies $\pm1$, giving a \textit{projectively flat} configuration.

\subsection{Octahedral Bianchi Constraint}
For every octahedral cell $\mathcal{O}$,
\[
\prod_{p\subset\partial\mathcal{O}}W(p)=+1,
\]
which forbids monopole--like defects and allows only even central flux.

\section{Electric Charge: U(1) Phase Projection}

\subsection{t’Hooft-Type Abelian Projection}
Given a local axis $m_i$, define
\[
u_{ij}=
\frac{\mathrm{Tr}\!\left(\frac{1+m_i\cdot\sigma}{2}\,U_{ij}\right)}
{ \big|\mathrm{Tr}\!\left(\frac{1+m_i\cdot\sigma}{2}\,U_{ij}\right)\big|}
=e^{i a_{ij}},
\qquad a_{ij}\in\mathbb{R}/2\pi\mathbb{Z}.
\]
The loop sum satisfies $\sum_{(ij)\in\ell}a_{ij}=2\pi n$.

\subsection{Charge Quantization}
Due to $\mathbb{Z}_{12}$ torsion, $a_{ij}\in(\pi/6)\mathbb{Z}$.
Hence each tetrahedral (``quark cell'') closure quantizes electric charge in units of $e/6$:
\[
Q_i=\frac{e}{6}s_i,\qquad 
s_i=\mathrm{sign}\Big(\mathrm{Tr}\big[q_i(\hat n_p\cdot\vec\sigma)q_i^{-1}(\hat m_p\cdot\vec\sigma)\big]\Big),
\]
\[
Q(\mathcal{T})=\frac{e}{6}\sum_{i\in\mathcal{T}}s_i
 \in\{-\tfrac{2}{3},-\tfrac{1}{3},0,\tfrac{1}{3},\tfrac{2}{3}\}e.
\]

\section{Color Charge: Dihedral Class and SU(3) Embedding}

\subsection{Plaquette Phase Quadruples}

For a square plaquette, let 
\[
(a,b,c,d)\in((\pi/6)\mathbb{Z}/2\pi\mathbb{Z})^4,\qquad 
a+b+c+d\equiv0\ (\text{mod }2\pi).
\]
Write integer form $k_i=(6/\pi)a_i\in\mathbb{Z}_{12}$ with
$k_1+k_2+k_3+k_4\equiv0$ (mod 12).

\subsection{Cyclic Words and $D_4$ Action}

Define the cyclic word space
\[
\mathcal{W}=\bigl\{[k_1,k_2,k_3,k_4]\mid k_i\in\mathbb{Z}_{12},\
\sum k_i\equiv0\bigr\}/\text{(cyclic shift)}.
\]
The square dihedral group $D_4$ acts on $\mathcal{W}$ by rotation and reflection.

\textbf{Quark--Plaquette Assumption:}
Only plaquettes with all four $k_i$ distinct are admissible (degenerate sets are colorless).

\textbf{Theorem.}
For distinct $k_i$, the number of orbits under $D_4$ is $3$.
By Burnside’s lemma, since nontrivial elements fix no configurations,
$|\mathcal{W}/D_4|=\frac{1}{8}\times24=3$.

\subsection{Color Charge Function}
Define
\[
\kappa:\ \mathcal{W}_{\mathrm{adm}}/D_4 \longrightarrow
   \{\pm r,\pm g,\pm b,0\},
\]
assigning the three orbits $\mathcal{O}_1,\mathcal{O}_2,\mathcal{O}_3$
to $\{r,g,b\}$ by convention.
Orientation reversal $\iota:[k_1,k_2,k_3,k_4]\mapsto[k_1,k_4,k_3,k_2]$
acts as
$\kappa(\iota[w])=-\kappa([w])$ (anticolor).
If $k_i$ are not distinct, $\kappa=0$.

\subsection{Closure Sector and Admissible Sets}
Within the modular-12 closure sector
($k_1+k_2+k_3+k_4\equiv0$),
the representative integer quadruples ($k_i\in\{-5,\ldots,6\}$) number $42$.
Imposing 3D octahedral consistency requires $0\in K=\{0,x,y,z\}$,
leaving $14$ equivalence classes after symmetries are factored out.

\subsection{SU(3) Static Embedding}
Associate $\{r,g,b\}$ with the fundamental weights $\{\omega_1,\omega_2,\omega_3\}$:
\[
\omega_1+\omega_2+\omega_3=0,\qquad
\alpha_1=\omega_1-\omega_2,\ \alpha_2=\omega_2-\omega_3,\ \alpha_3=\alpha_1+\alpha_2.
\]
Orientation reversal corresponds to $(t_3,t_8)\mapsto-(t_3,t_8)$ or $\omega\mapsto-\omega$.
When the $14$ admissible $(a,b,c)$ triples are projected onto the Cartan plane $(T_3,T_8)$,
their coordinates lie precisely on integer lattice points in the $(\omega_1,\omega_2)$ basis.

\section{Binding Rules and 3D Consistency}

\subsection{Meson}
A plaquette and its reverse orientation combine as
\[
\omega + (-\omega)=0,
\]
implying color neutrality.

\subsection{Baryon (Octahedral Closure)}
For three mutually orthogonal plaquettes (in $XY,YZ,ZX$ planes)
sharing edges to close an octahedron, the closure condition is
\[
K=\{0,x,y,z\},\qquad x+y+z\equiv0\pmod{12}.
\]
Arranged as
\[
XY:[0,x,y,z],\quad YZ:[0,y,z,x],\quad ZX:[0,z,x,y],
\]
all eight triangular faces close simultaneously.
When the three plaquette colors differ,
\[
\omega_1+\omega_2+\omega_3=0,
\]
forming a color singlet baryon.

\section{Global Topological Sectors}
On a 3-torus with noncontractible loops $C_x,C_y,C_z$,
\[
W_x,W_y,W_z\in\{\pm1\},\qquad
\mathcal{H}_{\mathrm{global}}\cong H^2(T^3,\mathbb{Z}_2)\cong\mathbb{Z}_2^3.
\]
These global $Z_2$ choices fix the background spin parity sector.

\section{Hierarchical Summary}
\begin{center}
\begin{tabular}{llll}
\hline
Level & Geometry & Gauge Group & Observable \\
\hline
Spin & triangular/square loops & $SU(2)$ (center $Z_2$) & $W(\ell)=\pm1$ \\
Charge & internal phase (Hopf fiber) & $U(1)$ & $a_{ij}\in(\pi/6)\mathbb{Z}$, $e/6$ \\
Color & plaquette cyclic orbit & $SU(3)$ (Cartan $T^2$) & $\kappa\in\{\pm r,\pm g,\pm b,0\}$ \\
Consistency & octahedral closure & $Z_2$ Bianchi & even flux, 14 reps \\
\hline
\end{tabular}
\end{center}

\section{Logical Cohesion}
\begin{enumerate}
\item Common quantum unit: $\pi/6$ for both charge and color.
\item Common antisymmetry: spin parity ($Z_2$) $\leftrightarrow$ color reversal (order 2).
\item Common closure rule: $\sum k_i\equiv0$ (mod 12) governs charge loops and color plaquettes alike.
\item Observables are not traces but combinatorial orbit invariants—topological markers of degenerate sectors.
\end{enumerate}

\section{Physical Interpretation}
\begin{itemize}
\item Fractional charges are pinned by lattice $\mathbb{Z}_{12}$ torsion, stabilized by the $Z_2$ Bianchi constraint.
\item Color degrees of freedom survive as combinatorial invariants $\kappa$ within degenerate energy layers.
\item Meson and baryon color neutrality arise geometrically from plaquette and octahedral closure.
\item Global $Z_2^3$ sectors may act as parity backgrounds controlling defect condensation.
\end{itemize}

\section{Conclusion}
The FCC lattice provides a natural discrete arena where the SU(2) spin layer, its internal U(1) phase, and the combinatorial SU(3) color structure share a single quantization scale and closure rule.
Spin is \textit{projectively flat}, charge arises from the internal U(1) projection, and color emerges from the $D_4$ cyclic orbits of $\mathbb{Z}_{12}$ plaquette phases.
Baryon and meson neutrality follow from the same modular--12 closure.
Thus, fractional charge and color confinement appear as geometric consequences of FCC lattice topology.

\end{document}
